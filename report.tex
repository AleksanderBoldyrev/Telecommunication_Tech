\include{settings}

\begin{document}	% начало документа

% Титульная страница
\include{titlepage}

% Содержание
\include{ToC}


\section{Цель работы}
Целью данной работы является приобретение навыков генерации и визуализации простых сигналов в среде MatLab, а также применение разложения Фурье для построения спектра сигналов.

\section{Постановка задачи}
Задачей работы является промоделировать сигналы в командном окне MATLAB и в среде Simulink из Главы 3, сс. 150–170 справочного пособия и получить их спектральное представление.

\section{Теоретическая информация}
\subsection{Понятие сигналов как векторов отсчетов функций}
В среде  MatLab любой сигнал может быть удобно представлен как вектор дискретных отсчетов некоторой функции (аналоговой или дискретной, периодической или апериодической и т.д.). Графичеки это представляется в виде зависимости значений этого вектора от значений вектора отсчетов времени. Второй вектор удобно формировать как возрастающую последовательность чисел, шаг между которыми есть величина, равная периоду дискретизации.

Таким образом, определив вектор отсчетов времени и некоторые константы, необходимые для представления вида сигнала в математической формуле, такие как амплитуда колебаний, частота колебаний и фаза, мы может задать вектор значений функции в известные нам моменты времени для дальнейшего построения графика. Делается это путем использования известных математических законов и встроенных в MatLab функций генерации специальных сигналов.

\subsection{Затухающие сигналы}
Затухание обычного гармонического сигнала можно получить путем его домножения на убывающую экспоненциальную функцию, как показано ниже на \ref{pic:t1}:
\begin{equation}
	s2 = exp^{-\alpha t}  s1
\end{equation}
где s1 - гармонический сигнал

\subsection{Одиночные импульсы}
Встроенная функция rectpuls имеет следующий вид:
\begin{equation}
	y = 
	\begin{cases}
		1, \ \ \ \  -\frac{width}{2} \leqslant t \leqslant \frac{width}{2}
  		\\
		0, \ \ \ \ t < -\frac{width}{2}, t > \frac{width}{2}
	\end{cases}
\end{equation}
где y-возвразаемое значение, t-вектор значений времени, сгенерированный заранее, width-ширин (длительность) импульса.

Встроенная функция tripuls имеет следующий вид:
\begin{equation}
	y =
	\begin{cases}
		\frac{2t+width}{width(skew+1)}, \ \ \ \  -\frac{width}{2} \leqslant t < \frac{width*skew}{2}
		\\
		\frac{2t-width}{width(skew-1)}, \ \ \ \  \frac{width*skew}{2} \leqslant t < \frac{width}{2}
		\\
		0, \ \ \ \ \ \ \ \ \ \ \ \ \ \ \ \ \ \  |t| > \frac{width}{2}
	\end{cases}
\end{equation}
где параметр skew - коэффициент ассимметрии импульса (по-умолчанию равен 0), а другие параметры имеют те же значения.

\subsection{Ограниченная полоса частот}
Для формирования сигнала, имещего ограниченный спектр, используется функция sinc:
\begin{equation}
	y = \frac{sin(\pi x)}{\pi x}
\end{equation}
Спектр сигнала в этом случае имеет прямоугольный вид:
\begin{equation}
	y =
	\begin{cases}
		1, \ \ \ \ |\omega| < \pi
		\\
		0, \ \ \ \ |\omega| > \pi
	\end{cases}
\end{equation}

\subsection{Гауссов радиоимпульс}
Функция для получения отсчетов радиоимпульса имеет фнутри себя следую математичскию формулу:
\begin{equation}
	y = exp^{-\alpha t^2} cos(2\pi f_ct)
\end{equation}
А спектр такого сигнала можно получить путем рпеобразования Фурье, формула которого представлена на \ref{pic:t7_2} ниже:
\begin{equation}
	S(\omega) = \frac{1}{2} \sqrt{\frac{\pi}{\alpha}} \Bigg( exp^{-\frac{(\omega + 2\pi f_c)^2}{4\alpha}} +  exp^{-\frac{(\omega - 2\pi f_c)^2}{4\alpha}}  \Bigg)
\end{equation}

\subsection{Дирихле}
Функция Дирихле описывается формулой:
\begin{equation}
	diric_n(x) = \frac{sin(n\frac{x}{2})}{n sin(\frac{x}{2})}
\end{equation}
где n - целое положительное число.

Функцию Дирихле еще называют периодический sinc функцией.
При нечетном / четном значении параметра n функция приобретает вид:
\begin{equation}
	diric_n(x) = \sum \limits_{k = -\infty}^{\infty} sinc \bigg( n \bigg( \frac{t}{2\pi} - k \bigg) \bigg)
\end{equation}

\begin{equation}
	diric_n(x) = \sum \limits_{k = -\infty}^{\infty} (-1)^k sinc \bigg( n \bigg( \frac{t}{2\pi} - k \bigg) \bigg)
\end{equation}

\subsection{Математические законы изменения мгновенной частоты}
В данной работе рассматриваются 3 закона - линейный, квадратичный и логарифмический. Формулы этих законов представлены на \ref{pic:t9_1} - \ref{pic:t9_3}:
\begin{equation}
	f(t) = f_0 + \beta t, \text{ где } \beta = \frac{f_1 - f_0}{t_1}
\end{equation}
\begin{equation}
	f(t) = f_0 + \beta t^2, \text{ где } \beta = \frac{f_1 - f_0}{t_1^2}
\end{equation}
\begin{equation}
	f(t) = f_0 + e^{\beta t}, \text{ где } \beta = \frac{ln(f_1 - f_0)}{t_1}
\end{equation}
Стоит отметить, что логарифмический закон имеет экспоненциальную зависимость частоты от времени.

\subsection{Преобразование Фурье}
Для нахождение спектра сигнала чаще всего применяют разложение функции в ряд Фурье, или же преобразование Фурье.
Формула прямого преобразования Фурье выглядит следующим образом:
\begin{equation}
	S(\omega) = \int \limits_{-\infty}^{\infty} s(t)e^{-j\omega t} dt
\end{equation}

Обратное преобразование Фурье строится по следующей формуле:
\begin{equation}
	s(t) = \frac{1}{2\pi} \int \limits_{-\infty}^{\infty} S(\omega)e^{j\omega t} d\omega
\end{equation}

\subsection{Корреляция}
Для нахождения синхропосылки в сигнале часто используется метод взаимной корреляции. Значение корреляции двух векторов x и y строится по формуле:
\begin{equation}
	R = \frac {1}{N} \sum \limits_{i=1}^{N} x_i * y_i
\end{equation}
где N - длина векторов х и y. Если искомая посылка у короче передаваемого вектора х, то она дополняется нулями до необходимой длины.

Для определения позиции синхропосылки в передаче необходимо сдвигать вектор у пошагово, на каждом шаге высчитывая значение корреляции и, таким образом, получая вектор значений корреляции. Максимальное значение этого вектора будет соответствовать сдвигу, при котором была найдена искомая посылка или же максимально похожая на нее часть вектора.

Для ускорения вычисления корреляции, особенно в больших посылках, применим метод быстрой корреляции:
\begin{equation}
	R = \frac{1}{N} F_D^-1 [X^* * Y]
\end{equation}
Где $X^*$ - комплексно-сопряженный вектор от вектора преобразования Фурье от посылки х, Y - результат преобразования Фурье от вектора искомой синхропосылки,$ F_D^-1$ - Обратное преобразование Фурье.

Данная формула позволяет найти вектор значений взаимной корреляции двух векторов быстрее, нежели обычный алгоритм нахождения корреляции.

\section{Ход работы}

\subsection{Генерация затухающего гармонического сигнала}

\lstinputlisting[
	label=code:m1,
	caption={Код в MatLab},% для печати символ '_' требует выходной символ '\'
]{m1.m}
\parindent=1cm
Здесь представлен код программы, генерирующей затухающий сигнал и выводящий на экран 4 различных графика этого сигнала.

\begin{figure}[H]
	\begin{center}
		\includegraphics[scale=0.7]{g_1}
		\caption{Графики затухающего сигнала} 
		\label{pic:g_1} % название для ссылок внутри кода
	\end{center}
\end{figure}
На первом графике виден обычный вид затухающего гармонического сигнала, построенный средой MatLab по дискретным отсчетам. Второй график представляет из себя точки того же сигнала, соответствующие дискретным отсчетам. Третий график (stem) представляет собой те же точки, но в виде «лепестков» - как некоторые значения, отклоненные от нулевого. Четвертый график (stairs) — ступенчатый графк.

\begin{figure}[H]
	\begin{center}
		\includegraphics[scale=0.7]{spec1}
		\caption{Спектр затухающего сигнала} 
		\label{pic:spec1} % название для ссылок внутри кода
	\end{center}
\end{figure}
Спектр представленного выше сигнала получен с помощью разложение в ряд Фурье.

\subsection{Многоканальный сигнал}

\lstinputlisting[
	label=code:m2,
	caption={Код в MatLab},% для печати символ '_' требует выходной символ '\'
]{m2.m}
\parindent=1cm
Данный код генерирует сразу несколько сигналов, записываемых в одну матрицу, различающихся по частоте.

\begin{figure}[H]
	\begin{center}
		\includegraphics[scale=0.7]{g_2}
		\caption{Графики синусоидальных сигналов различных частот: 60, 120 и 140 Гц} 
		\label{pic:g_2} % название для ссылок внутри кода
	\end{center}
\end{figure}
На данном графике видно несколько гармонических сигналов, различающихся по частоте.

\begin{figure}[H]
	\begin{center}
		\includegraphics[scale=0.7]{spec2}
		\caption{Спектры синусоидальных сигналов различных частот: 60, 120 и 140 Гц} 
		\label{pic:spec2} % название для ссылок внутри кода
	\end{center}
\end{figure}
На этом рисунке видны спектры данных синусоид. Линии спектра сигнала с более высокой частотой распологаются ближе к нулю.

\subsection{Кусочные зависимости}

\lstinputlisting[
	label=code:m3,
	caption={Код в MatLab},% для печати символ '_' требует выходной символ '\'
]{m3.m}
\parindent=1cm
Данный код генерирует и выводит на экран односторонний экспоненциальный импульс, прямоугольный импульс и несимметричный треугольный импульс согласно заданным параметрам.

\begin{figure}[H]
	\begin{center}
		\includegraphics[scale=0.7]{g_3_1}
		\caption{Экспоненциальный импульс} 
		\label{pic:g_3_1} % название для ссылок внутри кода
	\end{center}
\end{figure}
\begin{figure}[H]
	\begin{center}
		\includegraphics[scale=0.7]{g_3_2}
		\caption{Прямоугольный импульс} 
		\label{pic:g_3_2} % название для ссылок внутри кода
	\end{center}
\end{figure}
\begin{figure}[H]
	\begin{center}
		\includegraphics[scale=0.7]{g_3_3}
		\caption{Несимметричный треугольный импульс} 
		\label{pic:g_3_3} % название для ссылок внутри кода
	\end{center}
\end{figure}
На рисунках  \ref{pic:g_3_1} — \ref{pic:g_3_3} представлены графики сгенерированных сигналов, выведенных с помощью стандартной функции построения графиков в MatLab.

\begin{figure}[H]
	\begin{center}
		\includegraphics[scale=0.7]{spec3_1}
		\caption{Спектр экспоненциального импульса} 
		\label{pic:spec3_1} % название для ссылок внутри кода
	\end{center}
\end{figure}
\begin{figure}[H]
	\begin{center}
		\includegraphics[scale=0.7]{spec3_2}
		\caption{Спектр прямоугольного импульса} 
		\label{pic:spec3_2} % название для ссылок внутри кода
	\end{center}
\end{figure}
\begin{figure}[H]
	\begin{center}
		\includegraphics[scale=0.7]{spec3_3}
		\caption{спектр несимметричного треугольного импульса} 
		\label{pic:spec3_3} % название для ссылок внутри кода
	\end{center}
\end{figure}
На рисунках  \ref{pic:spec3_1} — \ref{pic:spec3_3} представлены спектры сигналов \ref{pic:g_3_1} — \ref{pic:g_3_3}.

\subsection{Прямоугольный импульс}

\lstinputlisting[
	label=code:m5,
	caption={Код в MatLab},% для печати символ '_' требует выходной символ '\'
]{m5.m}
\parindent=1cm
Данный сигнал представлен конкатенацией двух разнополярных прямоугольных импульсов, с использованием встроенных функций.

\begin{figure}[H]
	\begin{center}
		\includegraphics[scale=0.7]{g_4}
		\caption{Прямоугольные импульсы} 
		\label{pic:g_4} % название для ссылок внутри кода
	\end{center}
\end{figure}
На данном рисунке представлен график прямоугольных импульсов.

\begin{figure}[H]
	\begin{center}
		\includegraphics[scale=0.7]{spec4}
		\caption{Спектр прямоугольных импульсов} 
		\label{pic:spec4} % название для ссылок внутри кода
	\end{center}
\end{figure}
Спектр прямоугольного импульса получен с помощью преобразования Фурье.

\subsection{Трапецевидный импульс}

\lstinputlisting[
	label=code:m6,
	caption={Код в MatLab},% для печати символ '_' требует выходной символ '\'
]{m6.m}
\parindent=1cm
Данный сигнал генерируется разностью двух треугольных импульсов, с использованием встроенной функции tripuls.

\begin{figure}[H]
	\begin{center}
		\includegraphics[scale=0.7]{g_5}
		\caption{Трапецевидный импульс} 
		\label{pic:g_5} % название для ссылок внутри кода
	\end{center}
\end{figure}
На данном рисунке представлен вид трапецевидного импульса в среде MatLab.

\begin{figure}[H]
	\begin{center}
		\includegraphics[scale=0.7]{spec5}
		\caption{Спектр трапецевидного импульса} 
		\label{pic:spec5} % название для ссылок внутри кода
	\end{center}
\end{figure}
На рисунке представлен спектр трапецевидного импульса.

\subsection{Сигнал с ограниченной полосой частот}

\lstinputlisting[
	label=code:m7,
	caption={Код в MatLab},% для печати символ '_' требует выходной символ '\'
]{m7.m}
\parindent=1cm
Данный код генерирует сигнал, у которого спектр ограничен по частоте. Затем выводится и сам спектр данного сигнала:

\begin{figure}[H]
	\begin{center}
		\includegraphics[scale=0.7]{g_6_1}
		\caption{Сигнал с ограниченным спектром} 
		\label{pic:g_6_1} % название для ссылок внутри кода
	\end{center}
\end{figure}
\begin{figure}[H]
	\begin{center}
		\includegraphics[scale=0.7]{g_6_2}
		\caption{Ограниченный спектр ограниченного сигнала} 
		\label{pic:g_6_2} % название для ссылок внутри кода
	\end{center}
\end{figure}
Спектр сигнала получен с помощью функции sinc.

\subsection{Гауссов радиоимпульс}

\lstinputlisting[
	label=code:m8,
	caption={Код в MatLab},% для печати символ '_' требует выходной символ '\'
]{m8.m}
\parindent=1cm
Данный код генерирует Гауссов радиоимпульс с помощью встроенной функции gauspuls, а затем находит спектр этого сигнала, выражая его в дБ.

\begin{figure}[H]
	\begin{center}
		\includegraphics[scale=0.7]{g_7_1}
		\caption{Гауссов радиоимпульс} 
		\label{pic:g_7_1} % название для ссылок внутри кода
	\end{center}
\end{figure}
\begin{figure}[H]
	\begin{center}
		\includegraphics[scale=0.7]{g_7_2}
		\caption{Амплитудный спектр радиоимпульса} 
		\label{pic:g_7_2} % название для ссылок внутри кода
	\end{center}
\end{figure}
На графике спектра также отмечены расчетные границы этого спектра.

\subsection{Последовательности импульсов}

\lstinputlisting[
	label=code:m9,
	caption={Код в MatLab},% для печати символ '_' требует выходной символ '\'
]{m9.m}
\parindent=1cm
Данный код генерирует треугольные импульсы с заданными амплитудами, через заданные промежутки времени.

\begin{figure}[H]
	\begin{center}
		\includegraphics[scale=0.7]{g_8_1}
		\caption{Треугольные импульсы} 
		\label{pic:g_8_1} % название для ссылок внутри кода
	\end{center}
\end{figure}
\begin{figure}[H]
	\begin{center}
		\includegraphics[scale=0.7]{spec8_1}
		\caption{Спектр треугольных импульсов} 
		\label{pic:spec8_1} % название для ссылок внутри кода
	\end{center}
\end{figure}
На рисунках представлены - треугольные импульсы, сгенерированные с помощью встроенной функции, (\ref{pic:g_8_1}) и спектр этого сигнала (\ref{pic:spec8_1}).

\lstinputlisting[
	label=code:m10,
	caption={Код в MatLab},% для печати символ '_' требует выходной символ '\'
]{m10.m}
\parindent=1cm
Данный код генерирует и выводит гармонические импульсы.

\begin{figure}[H]
	\begin{center}
		\includegraphics[scale=0.7]{g_8_2}
		\caption{Гармонические импульсы} 
		\label{pic:g_8_2} % название для ссылок внутри кода
	\end{center}
\end{figure}
Данные импульсы сгенерированы функцией pulstran из вектора отсчетов одиночного импульса.
\begin{figure}[H]
	\begin{center}
		\includegraphics[scale=0.7]{spec8_2}
		\caption{Спектр гармонических импульсов} 
		\label{pic:spec8_2} % название для ссылок внутри кода
	\end{center}
\end{figure}
На рисунке представлен спектр гармонических импульсов.


\subsection{Генерация периодических сигналов}

\lstinputlisting[
	label=code:m11,
	caption={Код в MatLab},% для печати символ '_' требует выходной символ '\'
]{m11.m}
\parindent=1cm
Данная программа создает и выводит на экран периодически повторяющиеся прямоугольные сигналы, создаваемые с помощью функции square.

\begin{figure}[H]
	\begin{center}
		\includegraphics[scale=0.7]{g_9_1}
		\caption{Периодические прямоугольные импульсы} 
		\label{pic:g_9_1} % название для ссылок внутри кода
	\end{center}
\end{figure}
Импульсы обладают одинаковой длительностью и временем паузы между ними, что можно увидеть более отчетливо, если увеличить частоту дискретизации.
\begin{figure}[H]
	\begin{center}
		\includegraphics[scale=0.7]{spec9_1}
		\caption{Спектр прямоугольных импульсов} 
		\label{pic:spec9_1} % название для ссылок внутри кода
	\end{center}
\end{figure}

\lstinputlisting[
	label=code:m12,
	caption={Код в MatLab},% для печати символ '_' требует выходной символ '\'
]{m12.m}
\parindent=1cm
Эта программа, используя функцию sawtooth, создает импульсы треугольной формы с заданными параметрами.

\begin{figure}[H]
	\begin{center}
		\includegraphics[scale=0.7]{g_9_2}
		\caption{Треугольные импульсы sawtooth} 
		\label{pic:g_9_2} % название для ссылок внутри кода
	\end{center}
\end{figure}
\begin{figure}[H]
	\begin{center}
		\includegraphics[scale=0.7]{spec9_2}
		\caption{Спектр прямоугольных импульсов} 
		\label{pic:spec9_2} % название для ссылок внутри кода
	\end{center}
\end{figure}

\subsection{Функция Дирихле}

\lstinputlisting[
	label=code:m13,
	caption={Код в MatLab},% для печати символ '_' требует выходной символ '\'
]{m13.m}
\parindent=1cm
Программа использует встроенную функцию diric для создания выборки из функции Дирихле с четным и нечетным значением параметра.

\begin{figure}[H]
	\begin{center}
		\includegraphics[scale=0.7]{g_10_1}
		\caption{Функция Дирихле с параметром равным 7} 
		\label{pic:g_10_1} % название для ссылок внутри кода
	\end{center}
\end{figure}
\begin{figure}[H]
	\begin{center}
		\includegraphics[scale=0.7]{g_10_2}
		\caption{Функция Дирихле с параметром равным 8} 
		\label{pic:g_10_2} % название для ссылок внутри кода
	\end{center}
\end{figure}
Видно, что нечетный параметр обеспечивает однонаправленные импульсы, а большее значение параметра увеличивает частоту колебаний.

\begin{figure}[H]
	\begin{center}
		\includegraphics[scale=0.7]{spec10_1}
		\caption{Спектр функции Дирихле с параметром равным 7} 
		\label{pic:spec10_1} % название для ссылок внутри кода
	\end{center}
\end{figure}
\begin{figure}[H]
	\begin{center}
		\includegraphics[scale=0.7]{spec10_2}
		\caption{Српектр функция Дирихле с параметром равным 8} 
		\label{pic:spec10_2} % название для ссылок внутри кода
	\end{center}
\end{figure}

\subsection{Сигнал с меняющейся частотой}

\lstinputlisting[
	label=code:m14,
	caption={Код в MatLab},% для печати символ '_' требует выходной символ '\'
]{m14.m}
\parindent=1cm
Эта программа с помощью функции chirp генерирует колебания, мгновенная частота которых изменяется согласно выбранной функции. В данном примере рассмотрены 3 таких функции — линейная, квадратичная и логарифмическая. На экран выводятся спектрограммы этих сигналов — зависимость мгновенного амплитудного спектра от времени.

\begin{figure}[H]
	\begin{center}
		\includegraphics[scale=0.7]{g_11_1}
		\caption{Спектрограмма линейной функции chirp} 
		\label{pic:g_11_1} % название для ссылок внутри кода
	\end{center}
\end{figure}
\begin{figure}[H]
	\begin{center}
		\includegraphics[scale=0.7]{g_11_2}
		\caption{Спектрограмма квадратичной функции chirp} 
		\label{pic:g_11_2} % название для ссылок внутри кода
	\end{center}
\end{figure}
\begin{figure}[H]
	\begin{center}
		\includegraphics[scale=0.7]{g_11_3}
		\caption{Спектрограмма логарифмической функции chirp} 
		\label{pic:g_11_3} % название для ссылок внутри кода
	\end{center}
\end{figure}
На рисунках \ref{pic:g_11_2}, \ref{pic:g_11_2} и \ref{pic:g_11_3} показаны спектрограммы, наглядно демонстрирующие характер изменения мгновенной частоты сигнала во времени.

\subsection{Сравнение методов корреляции}
В качестве исходного примера была взята задача нахождения синхропосылки 101 в сигнале 0001010111000010.
\lstinputlisting[
	label=code:mCorr,
	caption={Код в МатЛаб},% для печати символ '_' требует выходной символ '\'
]{mCorr.m}
\parindent=1cm
Перед началом вычисления корреляции синхропосылка была изменена - (101) на (1-11) для улучшения качества этой посылки с целью ее более надежного нахождения в посылке, а затем посылка была дополнена нулями для совпадения длин двух векторов. 
Производились два рассчета корреляции - обычным алгоритмом и быстрым алгоритмом с фиксированием времени выполнения.
Оба алгоритма справились с нахождением синхропосылки, которая была найдена в сигнале дважды - по смещению относительно первого бита на +3 и +5. 
Первый алгоритм выполнил задачу за 0,124 мс, в то время как второй выполнил задачу за 0,041 мс. Можно сделать вывод о том, что алгоритм быстрой корреляции намного быстрее стандартного.

\section{Выводы}

В работе исследованы методы генерации и визуализации различных сигналов в среде MatLab. 

Рассмотрены различные виды сигналов - детерминиированные сигналы, периодические колебания и сигналы, полученные на их основе, сигналы, представляющие из себя единичные импульсы различной формы, конечные и бесконечные сигналы. Так же были получены и построены спектры сигналов с помощью преобразования Фурье, имеющего реализацию в среде MatLab.
Так же, были исследованы два метода подсчета корреляционной функции. Даже на простом и коротком примере быстрый алгоритм оказывается во много раз быстрее обычного алгоритма.

В качестве применений преобразования Фурье можно отметить следующие - фильтрация сигнала от шумов и помех, что представляется возможным, исходя из наличия паразитных гармоник в спектре; определение наличия полезного сигнала в случайном шуме; возможность иметь представление о присутствующих в сигнале частотах, равно как и определение факта апериодичности, конечности, дискретности искомого сигнала по его спектру; коррекция параметров сигнала путем увеличения/уменьшения амплитуды отдельных гармоник и т.д.
\end{document}

